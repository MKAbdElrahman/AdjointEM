\documentclass{tufte-handout}
\title{ Adjoint Computational Electromagnetics \\ 
Eigenvalue Problems	}
\author[mohamedkamal]{Mohamed Kamal Abd Elrahman \\
6 October, Giza,\\ Egypt}

\usepackage{amsmath}  % extended mathematics
\usepackage{graphicx} % allow embedded images
\setkeys{Gin}{width=\linewidth,totalheight=\textheight,keepaspectratio}
\graphicspath{{graphics/}} % set of paths to search for images
\usepackage{amsmath}  % extended mathematics
\usepackage{booktabs} % book-quality tables
\usepackage{units}    % non-stacked fractions and better unit spacing
\usepackage{multicol} % multiple column layout facilities
\usepackage{lipsum}   % filler text
\usepackage{fancyvrb} % extended verbatim environments
\fvset{fontsize=\normalsize}% default font size for fancy-verbatim environments

% Standardize command font styles and environments
\newcommand{\doccmd}[1]{\texttt{\textbackslash#1}}% command name -- adds backslash automatically
\newcommand{\docopt}[1]{\ensuremath{\langle}\textrm{\textit{#1}}\ensuremath{\rangle}}% optional command argument
\newcommand{\docarg}[1]{\textrm{\textit{#1}}}% (required) command argument
\newcommand{\docenv}[1]{\textsf{#1}}% environment name
\newcommand{\docpkg}[1]{\texttt{#1}}% package name
\newcommand{\doccls}[1]{\texttt{#1}}% document class name
\newcommand{\docclsopt}[1]{\texttt{#1}}% document class option name
\newenvironment{docspec}{\begin{quote}\noindent}{\end{quote}}% command specification environment

\begin{document}
	\maketitle

\begin{abstract}
	The sensitivities of eigenvalues and eigenvectors due to perturbations in the deign variables are derived. The focus here is on linear, isotropic, nonmagnetic and transparent materials \sidenote[1]{In transparent materials, both $\epsilon$ and $\mu$ are real and positive.}. 
\end{abstract}
 \section{Problem Formulation}
Maxwell's equation for  reads 
\begin{subequations}
	\begin{align}
	\nabla \times \mathbf{E}  &= -j \mu_0  \omega \mathbf{H}\\
	\nabla \times \mathbf{H}  &=   j \omega \epsilon_0 \epsilon(r) \mathbf{E}
	\end{align}
\end{subequations}
which can be combined into a single PDE \sidenote[2]{We will refer to this equation as the master equation.} 
\begin{equation}\label{master_eq}
\nabla \times \frac{1}{\epsilon} \nabla \times \mathbf{H}   = \left(  \frac{\omega}{c}\right)^2 \mathbf{H}
\end{equation}
Equation. \ref{master_eq} is a standard eigenvalue problem of the form,
\begin{equation}
A x = \lambda x
\end{equation}
where we have identified the operator $A = \nabla \times \nabla \times  $ and the eigenvector  $x = \mathbf{E}$ with associated eigenvalue $\lambda = \omega^2$. The operator $A$  can be shown to be hermitian and positive-semidefinite. The eigenmodes can be normalized
\begin{equation}
x^T x = 1
\end{equation}
\section{Sensitivity Derivatives (Nondegenerate Case)}
Our optimization problem can  be formulated as

\begin{equation}\label{opt_problem}
\begin{aligned}
\min_{p} \quad & g(p,x,\lambda) \\
\textrm{s.t.} \quad &  A x = \lambda x \\ & 
  x^T x = 1
\end{aligned}
\end{equation}
where $p$ is a vector of design parameters and $g$ is the objective function we seek to minimize. To develop the eign-adjoint variable method, we first define the Lagrangian  as
\begin{equation}\label{lagranian}
L=  g(p,x,\lambda) + \kappa^T(A- \lambda)x + \sigma (1-x^T x) 
\end{equation} 
 where $\sigma$ and  $\kappa$  are the Lagrange multipliers, for eigenvalues and eigenvectors, respectively \sidenote[3]{We will choose the Lagrange multipliers in a clever way to cancel the annoying derivative terms later.}.  Direct differentiation of equation.\ref{lagranian} with respect to the design parameters 
 \begin{equation}
 \frac{d L}{d p} = \frac{\partial g}{\partial p} + \frac{\partial g}{\partial x} \frac{d x}{d p}+ \frac{\partial g}{\partial \lambda} \frac{d \lambda}{d p} + \kappa^T \left( \frac{d A}{d p}- \frac{d \lambda}{d p}\right)  x + \kappa^T \left( A- \lambda \right) \frac{d x}{d p } -2  \sigma \left( x^T \frac{d x}{d p } \right)  
 \end{equation}
 Collecting terms in $\frac{d \lambda}{d p }$ and $\frac{d x }{d p}$ 
 
  \begin{equation}
  \frac{d L}{d p} = \frac{\partial g}{\partial p} + \left(  \frac{\partial g}{\partial x} + \kappa^T (A-\lambda) -2 \sigma x^T \right)  \frac{d x}{d p}+ \left(   \frac{\partial g}{\partial \lambda}- \kappa^T x  \right)  \frac{d \lambda}{d p} + \kappa^T  \frac{d A}{d p}  x  
  \end{equation}
  
  Requiring $\kappa$ and $\sigma$ to satisfy the following equations,  
  \begin{equation}
 \kappa^T x   =  \frac{\partial g}{\partial \lambda}
  \end{equation}
  \begin{equation}\label{11}
  (A-\lambda) \kappa -2 \sigma x   = -\left(  \frac{\partial g}{\partial x} \right) ^T
  \end{equation}
causes the  implicit design variation of the
Lagrangian to vanish. The second equation is problematic: $A - \lambda$ is singular, with a null space of $x$. So, we  also require $\sigma$  to make the solution orthogonal to the  the null space:
$x^T \left( \left(  \frac{\partial g}{\partial x} \right) ^T -2 \sigma x \right)  = 0  $. Then, $\sigma = \frac{x^T \left( \frac{\partial g}{\partial x} \right)^T }{2}$

\end{document}