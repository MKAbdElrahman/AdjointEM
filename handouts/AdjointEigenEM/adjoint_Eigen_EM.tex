\documentclass{tufte-handout}
\title{ Adjoint Computational Electromagnetics}
\author[mohamedkamal]{Mohamed Kamal Abd Elrahman \\
6 October, Giza,\\ Egypt}

\usepackage{amsmath}  % extended mathematics
\usepackage{graphicx} % allow embedded images
\setkeys{Gin}{width=\linewidth,totalheight=\textheight,keepaspectratio}
\graphicspath{{graphics/}} % set of paths to search for images
\usepackage{amsmath}  % extended mathematics
\usepackage{booktabs} % book-quality tables
\usepackage{units}    % non-stacked fractions and better unit spacing
\usepackage{multicol} % multiple column layout facilities
\usepackage{lipsum}   % filler text
\usepackage{fancyvrb} % extended verbatim environments
\fvset{fontsize=\normalsize}% default font size for fancy-verbatim environments

% Standardize command font styles and environments
\newcommand{\doccmd}[1]{\texttt{\textbackslash#1}}% command name -- adds backslash automatically
\newcommand{\docopt}[1]{\ensuremath{\langle}\textrm{\textit{#1}}\ensuremath{\rangle}}% optional command argument
\newcommand{\docarg}[1]{\textrm{\textit{#1}}}% (required) command argument
\newcommand{\docenv}[1]{\textsf{#1}}% environment name
\newcommand{\docpkg}[1]{\texttt{#1}}% package name
\newcommand{\doccls}[1]{\texttt{#1}}% document class name
\newcommand{\docclsopt}[1]{\texttt{#1}}% document class option name
\newenvironment{docspec}{\begin{quote}\noindent}{\end{quote}}% command specification environment

\begin{document}
	\maketitle

 \section{Problem Formulation}
Time harmonic coupled first order Maxwell's equation for a linear nonmagnetic and source-free medium could be written as 
\begin{subequations}
	\begin{align}
	\nabla \times \mathbf{E}  &= -j \mu_0  \omega \mathbf{H}\\
	\nabla \times \mathbf{H}  &=   j \omega \epsilon \mathbf{E}
	\end{align}
\end{subequations}
Which could be combined into a single second order PDE 
\begin{equation}
\nabla \times \nabla \times \mathbf{E}  = \mu_0  \epsilon_0 \omega^2    \epsilon_r \mathbf{E} = \left( \frac{\omega}{c} \right)^2     \epsilon_r \mathbf{E}
\end{equation}
This can be written in the standard generalized eigenvalue problem form.
\begin{equation}
A x = \lambda B x
\end{equation}
If the the following transformations have been made, $A = \nabla \times \nabla \times  $, $\mathbf{B} = \left( \frac{1}{c} \right)^2     \epsilon_r $ and  $x = \mathbf{E}$. The operators $A$ and $B$ can be shown to be positive-semidefinite and Hermitian. $B$ is even a Hermitian operator. It can be shown that $\omega$ is real, and that two solutions $x_i$ and $x_j$ with different frequencies satisfy an orthonormality relation:
\begin{equation}
x_i^T \epsilon x_j = \delta_{ij}
\end{equation}
Suppose we want to maximize the design objective $g(x,\lambda,\epsilon)$, the sensitivity of $g$ is 
\begin{equation}
\frac{d g}{d \epsilon} = \frac{\partial g}{\partial \epsilon} +  \frac{\partial g}{\partial x}  \frac{\partial x}{\partial \epsilon} +  \frac{\partial g}{\partial \lambda} \frac{\partial \lambda}{\partial \epsilon}
\end{equation} 

Taking the derivative of the eigenvalue equation
\begin{equation}
(A -\lambda B) \frac{\partial  x}{\partial \epsilon}   =  \frac{\partial \lambda}{\partial \epsilon} B x +  \lambda  \frac{\partial  B}{\partial \epsilon} x 
\end{equation}
Taking the derivative of the orthogonality equation
\end{document}