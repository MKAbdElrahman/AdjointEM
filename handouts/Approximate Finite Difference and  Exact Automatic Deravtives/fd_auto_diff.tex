\documentclass{tufte-handout}
\title{Sensitivity Calculations}
\author[mohamedkamal]{Mohamed Kamal Abd Elrahman \\
6 October, Giza,\\ Egypt}

\usepackage{amsmath}  % extended mathematics
\usepackage{graphicx} % allow embedded images
\setkeys{Gin}{width=\linewidth,totalheight=\textheight,keepaspectratio}
\graphicspath{{graphics/}} % set of paths to search for images
\usepackage{amsmath}  % extended mathematics
\usepackage{booktabs} % book-quality tables
\usepackage{units}    % non-stacked fractions and better unit spacing
\usepackage{multicol} % multiple column layout facilities
\usepackage{lipsum}   % filler text
\usepackage{fancyvrb} % extended verbatim environments
\fvset{fontsize=\normalsize}% default font size for fancy-verbatim environments

% Standardize command font styles and environments
\newcommand{\doccmd}[1]{\texttt{\textbackslash#1}}% command name -- adds backslash automatically
\newcommand{\docopt}[1]{\ensuremath{\langle}\textrm{\textit{#1}}\ensuremath{\rangle}}% optional command argument
\newcommand{\docarg}[1]{\textrm{\textit{#1}}}% (required) command argument
\newcommand{\docenv}[1]{\textsf{#1}}% environment name
\newcommand{\docpkg}[1]{\texttt{#1}}% package name
\newcommand{\doccls}[1]{\texttt{#1}}% document class name
\newcommand{\docclsopt}[1]{\texttt{#1}}% document class option name
\newenvironment{docspec}{\begin{quote}\noindent}{\end{quote}}% command specification environment

\begin{document}
	\maketitle
	\begin{abstract}
		\noindent 
	Sensitivities are used as measures of robustness for engineering systems. In many applications, one is interested about the system performance under small variations of a set of design parameters. In inverse device design, sensitivities guides the search within the space spanned by a set of design parameters.  
	\end{abstract}
\section{Numerical Differentiation}
\subsection{Finite Difference Method}
\subsection{Complex Step Method}
\section{Automatic Differentiation}

\subsection{Automatic Forward-Mode Differentiation}
\subsection{Automatic Reverse-Mode Differentiation}

\end{document}