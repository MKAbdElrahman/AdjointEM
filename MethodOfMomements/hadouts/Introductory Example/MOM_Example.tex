\documentclass{tufte-handout}

\title{Introduction to the Method of Moments}

\author[MK]{Mohamed Kamal AbdElrahman}

%\date{28 March 2010} % without \date command, current date is supplied

%\geometry{showframe} % display margins for debugging page layout

\usepackage{graphicx} % allow embedded images
  \setkeys{Gin}{width=\linewidth,totalheight=\textheight,keepaspectratio}
  \graphicspath{{graphics/}} % set of paths to search for images
\usepackage{braket}
\usepackage{amsmath}  % extended mathematics
\usepackage{booktabs} % book-quality tables
\usepackage{units}    % non-stacked fractions and better unit spacing
\usepackage{multicol} % multiple column layout facilities
\usepackage{lipsum}   % filler text
\usepackage{fancyvrb} % extended verbatim environments
  \fvset{fontsize=\normalsize}% default font size for fancy-verbatim environments

% Standardize command font styles and environments
\newcommand{\doccmd}[1]{\texttt{\textbackslash#1}}% command name -- adds backslash automatically
\newcommand{\docopt}[1]{\ensuremath{\langle}\textrm{\textit{#1}}\ensuremath{\rangle}}% optional command argument
\newcommand{\docarg}[1]{\textrm{\textit{#1}}}% (required) command argument
\newcommand{\docenv}[1]{\textsf{#1}}% environment name
\newcommand{\docpkg}[1]{\texttt{#1}}% package name
\newcommand{\doccls}[1]{\texttt{#1}}% document class name
\newcommand{\docclsopt}[1]{\texttt{#1}}% document class option name
\newenvironment{docspec}{\begin{quote}\noindent}{\end{quote}}% command specification environment



\begin{document}

\maketitle% this prints the handout title, author, and date

\begin{abstract}
This is an introduction the basic aspects of the Method of Moments (MOM).  
\end{abstract}
\section{Review}
Given $g(x)$, find $f(x)$ in the interval $ \Omega = [0,1]$ satisfying 
\begin{equation}
  \begin{aligned}
  -\frac{d^2 f}{d x^2} &=g(x),\quad \Omega\\
  f&=0  \quad \partial\Omega
  \end{aligned} 
\end{equation}
This is a boundary value problem of the form $L f = g$ for which 
\begin{align}
L =  -\frac{d^2 f}{d x^2}
\end{align}  
The operator $L$ is hermitian and positive-definite
\begin{equation}
\Braket{L f|g} = \Braket{f|L g}
\end{equation}  
 \begin{equation}
 \Braket{L f|f} \geq 0
 \end{equation}
 The inverse of operator $L$ can be obtained  with the help of standard Green's function techniques 
 \begin{equation}
 f(x) = L^{-1}(g) = \int_{0}^{1} G(x,x^{\prime})  g(x') dx'
 \end{equation}
 where G is the Green's function
 \begin{equation}
 G(x,x) = \begin{cases} 
 x(1-x^\prime) & x < x^\prime \\
 (1-x)x^\prime & x > x^\prime 
 \end{cases}
 \end{equation}
 The operator $L^{-1}$ is also Hermitian and positive-definite. Note that the boundary conditions must be specified for the domain of $L$, however, they are not required for $L^{-1}$ (Green functions already accounts for them).  
\end{document}