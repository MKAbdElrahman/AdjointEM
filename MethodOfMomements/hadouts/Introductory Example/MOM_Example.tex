\documentclass{tufte-handout}

\title{Introduction to the Method of Moments}

\author[MK]{Mohamed Kamal AbdElrahman}

%\date{28 March 2010} % without \date command, current date is supplied

%\geometry{showframe} % display margins for debugging page layout

\usepackage{graphicx} % allow embedded images
  \setkeys{Gin}{width=\linewidth,totalheight=\textheight,keepaspectratio}
  \graphicspath{{graphics/}} % set of paths to search for images
\usepackage{braket}
\usepackage{amsmath}  % extended mathematics
\usepackage{booktabs} % book-quality tables
\usepackage{units}    % non-stacked fractions and better unit spacing
\usepackage{multicol} % multiple column layout facilities
\usepackage{lipsum}   % filler text
\usepackage{fancyvrb} % extended verbatim environments
  \fvset{fontsize=\normalsize}% default font size for fancy-verbatim environments

% Standardize command font styles and environments
\newcommand{\doccmd}[1]{\texttt{\textbackslash#1}}% command name -- adds backslash automatically
\newcommand{\docopt}[1]{\ensuremath{\langle}\textrm{\textit{#1}}\ensuremath{\rangle}}% optional command argument
\newcommand{\docarg}[1]{\textrm{\textit{#1}}}% (required) command argument
\newcommand{\docenv}[1]{\textsf{#1}}% environment name
\newcommand{\docpkg}[1]{\texttt{#1}}% package name
\newcommand{\doccls}[1]{\texttt{#1}}% document class name
\newcommand{\docclsopt}[1]{\texttt{#1}}% document class option name
\newenvironment{docspec}{\begin{quote}\noindent}{\end{quote}}% command specification environment



\begin{document}

\maketitle% this prints the handout title, author, and date

\begin{abstract}
This is an introduction the basic aspects of the Method of Moments (MOM).  
\end{abstract}
\section{Review}
Given $g(x)$, find $f(x)$ in the interval $ \Omega = [0,1]$ satisfying 
\begin{equation}
  \begin{aligned}
  -\frac{d^2 f}{d x^2} &=g(x),\quad \Omega\\
  f&=0  \quad \partial\Omega
  \end{aligned} 
\end{equation}
This is a boundary value problem of the form $L f = g$ for which 
\begin{align}
L =  -\frac{d^2 f}{d x^2}
\end{align}  
The operator $L$ is hermitian and positive-definite
\begin{equation}
\Braket{L f|g} = \Braket{f|L g}
\end{equation}  
 \begin{equation}
 \Braket{L f|f} \geq 0
 \end{equation}
 The inverse of operator $L$ can be obtained  with the help of standard Green's function techniques 
 \begin{equation}
 f(x) = L^{-1}(g) = \int_{0}^{1} G(x,x^{\prime})  g(x') dx'
 \end{equation}
 where G is the Green's function
 \begin{equation}
 G(x,x) = \begin{cases} 
 x(1-x^\prime) & x < x^\prime \\
 (1-x)x^\prime & x > x^\prime 
 \end{cases}
 \end{equation}
 The operator $L^{-1}$ is also Hermitian and positive-definite. Note that the boundary conditions must be specified for the domain of $L$, however, they are not required for $L^{-1}$ (Green functions already accounts for them).  
 \section{Method of Moments}
 Consider the steady state equation
 \begin{equation}\label{ steady_state_eq}
 L f = g
 \end{equation}
 where $L$ is a linear operator, $g$ is a known excitation and the response $f$ is to be found. Let $f$ be expanded as a superposition of a set of basis functions $\Ket{v_i}$ with unknown expansion coefficients $c_i$
 \begin{equation}\label{superposition_eq}
f = \sum_i c_i \Ket{v_i} 
 \end{equation}
 For approximate solutions, (\ref{ steady_state_eq}) is truncated to a finite summation. Substituting (\ref{superposition_eq}) in (\ref{ steady_state_eq}), and using the linearity of $L$, we have
 \begin{equation} \label{stead_eq_expanded}
   \sum_i c_i L \Ket{v_i} = g
 \end{equation}  
 Taking the inner product of  (\ref{stead_eq_expanded}) with a set of test functions $w_m$, where $  m = 1,2, \dots $  
 \begin{equation}\label{linear_sys}
    \sum_i c_i  \braket{w_m |L v_i} = \Braket{w_m | g}
 \end{equation}
 once the basis and test functions are chosen, Equation (\ref{linear_sys}) becomes a standard linear system of algebraic equations. The procedure is similar to  the method of wighted residual commonly used for finite element methods (but instead of a differential operator we have an integral operator) and the particular choice $w_i = v_i$ is known as Galerkin method. In the finite element method, the basis (trial) and test functions have to be at least first order. In the moment method, basis and test function can be zero-order (pulse functions) because of the integral operator. 
 \section{The Collocation Method}
 The idea of the collocation method is to demand the satisfaction of (\ref{stead_eq_expanded}) at a number of discrete points in the region of interest. The collocation method can be also seen as taking Dirac delta functions as testing functions.
 \section{The Subdomain Collocation Method}
 The idea of  this approach is to choose zeroth-order test functions which exist only over subsections of the domain of $f$.   
 \section{Poisson Equation}
\end{document}