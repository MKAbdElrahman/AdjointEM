\documentclass[10pt]{beamer}

%\usetheme{metropolis}
\usepackage{appendixnumberbeamer}
\usepackage{amsmath}
\usepackage{booktabs}
\usepackage[scale=2]{ccicons}
\usepackage{framed,color}
\definecolor{shadecolor}{rgb}{1,.8,.3}
\usepackage{pgfplots}
\usepgfplotslibrary{dateplot}
\usepackage{breqn}

\usepackage{xspace}
\newcommand{\themename}{\textbf{\textsc{metropolis}}\xspace}
\usepackage{graphicx}
\usepackage{tikz}
\usefonttheme[onlymath]{serif}
\usetikzlibrary{%
	decorations.pathreplacing,%
	decorations.pathmorphing%
}
\title{Introduction to the Adjoint Method}

\author{
	Mohamed Kamal AbdElrahman \\
 \href{my_email}{s-mohamed.abdelrahman@zewilcity.edu.eg}}
\institute{University of Science and Technology \\[10pt] Zewail City}
\date{}
\begin{document}
\begin{frame}
	\titlepage
\end{frame}
\begin{frame}{Outline}
	\setbeamertemplate{section in toc}[sections numbered]
	\tableofcontents%[hideallsubsections]
\end{frame}
\section{Adjoint Sensitivity Analysis of Linear Systems}
\numberwithin{equation}{section}
\begin{frame}{Adjoint Sensitivity Analysis of Linear Systems}
	\begin{itemize}
		\item  Consider the Ordinary Differential Equation (ODE)
		\begin{equation}
a(t,z)\frac{d^2 \psi}{dt^2} + b(t,z)\frac{d \psi}{dt} +c(t,z) \psi = s(t) 
		\end{equation}
\item $z$ is a design parameter that we want to change in order to minimize the objective function $G$
\begin{equation}
G = \int^{T}_0  g(\psi,z) \, dt
\end{equation}
	\end{itemize}
\end{frame}

\begin{frame}{Adjoint Sensitivity Analysis of Linear Systems}
	\begin{itemize}
	\item  To minimize G, we  need the gradient with respect to the design parameter $z$.
	
	\item The  design parameter $z$ can then be updated with the steepest descent 
	\begin{equation}
	z = z - \alpha \frac{dG}{dz}
	\end{equation}
	
	\item Local descent involves iteratively determining a descent direction ($ \frac{dG}{dz}$) and	then taking a step $\alpha$ in that direction and repeating that process until convergence or some termination condition is met
	\end{itemize}
\end{frame}

\begin{frame}{Adjoint Sensitivity Analysis of Linear Systems}
	\begin{itemize}
		\item  Lets modify the objective function G
		\begin{equation}
G =  \int^{T}_0  g(\psi,z) \, dt + \int^{T}_0 \lambda(t) R \, dt 
		\end{equation}
		\item Where the residual $R = 0 =  a(t,z)\frac{d^2 \psi}{dt^2} + b(t,z)\frac{d \psi}{dt} +c(t,z) \psi- s(t) $
		\item The  equation is still valid and for any $\lambda(t)$, since we are multiplying $\lambda(t)$ by zero and we just added a zero to the whole equation. 
		\item $\lambda(t)$ is called the adjoint variable.
	\end{itemize}
\end{frame}

\begin{frame}{Adjoint Sensitivity Analysis of Linear Systems}
	\begin{itemize}
		\item Using the chain rule to obtain the objective function gradient:
		\begin{dmath}
\frac{dG}{dz} = \int^T_0 \frac{\partial g}{\partial z}
+ \frac{\partial g}{\partial \psi} \frac{\partial \psi}{\partial z} + \lambda \left[  \frac{\partial a}{\partial z} \ddot{\psi} + \\ a \frac{\partial \ddot{\psi}}{\partial z} + \frac{\partial b}{\partial z} \dot{\psi} + b \frac{\partial \dot{\psi}}{\partial z} + \frac{\partial c}{\partial z} \psi + c \frac{\partial \psi}{\partial z}\right] 
		\end{dmath} 
	\end{itemize}
\end{frame}
\end{document}